\documentclass{article}

\usepackage[affil-it]{authblk}
\usepackage{amsmath,amssymb}
\usepackage{bm}

\title{Recent progresses in Krylov subspace methods for solving complex symmetric linear systems}
\author{Xian-Ming Gu}
\affil{PhD student at Rijksuniversiteit Groningen and University of Electronic Science and Technology of China}
\date{February 2, 2015}

\begin{document}
\maketitle
Complex symmetric linear systems (CSLSs) with the following form
\begin{equation*}
A{\bm x} = {\bm b}, \quad\ A\neq A^H,\ \mathrm{but}\ A = A^T \in
\mathbb{C}^{n\times n},\quad {\bm b}\in \mathbb{C}^n
\end{equation*}
arise in many important applications such as numerical computations
in quantum chemistry, eddy current problems, modeling the waveguide
discontinuities and electromagnetic simulations. Hence, there is a
strong need for the fast solutions of complex symmetric linear systems.
During the past few years, a variety of specified Krylov subspace methods
(KSMs) for solving such systems are proposed and used, such as COCG, COCR,
QMR-SYM and BiCGCR methods.

In this talk, I will mainly revisit and focus on SCBiCG, which is also
known as one of methods for solving such linear system. SCBiCG can be
derived by substituting a matrix polynomial, which is expressed by the
complex conjugate coefficient matrix and initial residual vector, to
the initial shadow residual of the BiCG algorithm. Moreover, we clarify
that SCBiCG can be transformed to some methods which have been previously
proposed. Besides, in our talk we will prove that the preconditioned BiCGCR
is mathematically equivalent to preconditioned COCR in detail, and then give
an overview of the recent progress in other KSMs with suitable preconditioning
techniques for solving CSLSs. Finally, numerical experiments involving many
electromagnetic model problems are employed to investigate the convergence
behaviors of these solvers, and then some remarks on future research of this
topic will be also summarized.
\newline
\newline
\indent This is joint work with Ting-Zhu Huang, Liang Li, Tomohiro Sogabe,
Markus Clemens, Bruno Carpentieri, Hou-Biao Li.


\end{document}
