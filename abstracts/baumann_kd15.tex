\documentclass{article}

\usepackage[affil-it]{authblk}
\usepackage{amsmath,amssymb,mathtools}

\title{Krylov methods for shifted linear systems}
\author{Manuel Baumann}
\affil{PhD student at TU Delft}
\date{February 2, 2015}

\begin{document}
\maketitle
\setcounter{page}{4}
In my research, we focus on Krylov methods for so-called \textit{shifted linear systems} of the form
\begin{align}
\label{shifted}
 (A - \omega_k I) \mathbf{x}_k = \mathbf{b},
\end{align}
where $\{\omega_k\}_{k=1}^K \in \mathbb{C}$ is a sequence of distinct \textit{shifts}. During the last 20 years, almost every Krylov method has been adapted to solve \eqref{shifted} efficiently for many shifts. In my presentation, I will show you how multi-shift Krylov methods work and, afterwards, point to some more recent research questions like:
\begin{itemize}
 \item Can we allow multiple right-hand sides?
 \item Which preconditioners preserve the shifted structure?
 \item Can we apply restarting and nested algorithms?
 \item Can we benefit from deflation?
 \item Where do shifted systems arise in practice?
\end{itemize}


\end{document}
