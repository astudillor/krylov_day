\documentclass{article}

\usepackage{color}

\title{Dutch Seminar on Krylov subspace methods \\(Krylov day)}
\author{TU Delft SIAM Student Chapter}
\date{February 2, 2015}
\begin{document}
\maketitle
\section{Introduction}
Krylov subspace methods is an important tool which has been applied successfully to solve different problems 
in the Numerical Algebra. The Netherlands has been a pioneer country in the developing of Krylov method.
Methods like the Conjugate Gradient Squared (CGS), Bi-Conjugate Gradient Stabilized (BiCGSTAB), Nested GMRES (GMRESR), and the Induced Dimension Reduction method (IDR)  
are examples of Krylov methods developed in the Dutch universities. 
We are organizing The Dutch Seminar on Krylov subspace methods with the aim 
to bring together the young researcher to provide an
overview of the state-of-the-art of Krylov subspaces methods and encourage 
the collaboration among them.
% \newpage
% \section{Program:}
% This meeting is going to take place on February 2nd of 2015. Six to eight talks held in Delft University of Technology. 
% Tentative program:
% \begin{itemize}
% \item 9:30 to 10:00 Welcoming. 
% \item 10:00 to 10:30 
% \item 10:30 to 11:00 
% \item 11:00 to 11:15 Break.
% \item 11:15 to 11:45
% \item 11:45 to 12:15
% \item 12:15 to 13:30 Lunch.
% \item 13:30 to 14:00 
% \item 14:00 to 14:30 
% \item 14:30 to 14:45 Break 
% \item 14:45 to 15:15 
% \item 15:15 to 15:45 
% \item 15:45 to 16:15 
% \item 16:15 to 17:00 Closing
% \item 18:00 Dinner (Optional) 
% \end{itemize}
\newpage
\section{Program}
\begin{table}[h]
\begin{tabular}{|l|l|l|}
\hline
10:00 to 10:10 & Reinaldo & Welcoming \\
10:15 to 10:45 & Heiko & Preconditioning of Large-Scale Saddle Point Systems \\ 
 & & for Coupled Flow Problems\\
10:45 to 11:00 & & Coffee at EWI\\ 
11:00 to 11:30 & Manuel & Krylov methods for shifted linear systems\\
11:30 to 12:00 & Patrick &  Krylov subspace methods for matrix equations\\
 & & which include matrix functions\\
12:00 to 13:30 & &  Lunch at TUD Sports Center and campus tour\\
13:30 to 14:00 & & Talk 4\\
14:00 to 14:30 & & Talk 5\\
14:30 to 15:00 & & Talk 6\\
15:00 to 15:15 & &  Coffee at EWI \\
15:15 to 16:00 & & Talk 7 (``historic talk'' by professor)\\
16:00 to 16:10 & &  Closing\\
16:15 to 18:30 & & Beestenmarkt (first beer paid by SSC) \\
\hline
\end{tabular}
\end{table}

% \subsection{Students (min 5, max 6)}
% \begin{table}[h]
%  \centering
%  \begin{tabular}{|l|l|l|c|}
%  \hline
%  \textbf{Name} & \textbf{Topic} & \textbf{Uni} & \textbf{confirmed ?}\\
%  \hline
%  Manuel Baumann & Krylov methods for shifted linear systems & TU Delft & yes\\
%  Reinaldo Astudillo & IDR(s) for eigenvalue computation & TU Delft & yes \\
%  Yue & & TU Delft & no \\
%  Joost & & TU Delft & no \\
%  Ian Zwaan & & TUE & no \\
%  Sarah Graaf & & TUE & no \\
% 
%  \hline
%  \end{tabular}
% 
% \end{table}
% 
% \subsection{Staff (min 1, max 2)}
% \begin{itemize}
%  \item Kees Vuik, TUD
%  \item Martin van Gijzen, TUD
%  \item Domenico Lahaye, TUD
%  \item Peter Sonneveld, TUD
%  \item Mike Botchev, UT
%  \item Sleijpen, vd Vorst, UU
%  \item {\color{red}Michiel Hochstenbuch, TUE}
%  \item Jens-Peter Zemke, TUHH (?)
% \end{itemize}
% 
% \subsection{Budget}
% In total 200 \$ $\approx$ 150 EUR.
% \begin{itemize}
%  \item Coffe: free if from EWI
%  \item Cake: We bake one (= breakfast)
%  \item Lunch at aula: 8EUR pP
%  \item Beestenmarkt: Cake or beers for 5 EUR
% \end{itemize}

\section{Abstracts}
\end{document}
