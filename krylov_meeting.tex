\documentclass{article}

\usepackage{color}
\usepackage{hyperref}
\usepackage{pdfpages}


\title{\bf Student Krylov Day 2015}
\author{\href{http://sscdelft.github.io/activities/2015/02/02/krylov-day.html}{SIAM Student Chapter Delft}}
\date{February 2, 2015}
\begin{document}
\maketitle
\begin{figure}[h]
 \centering
 \includegraphics{220px-Alexey_Krylov_1910s.JPG}
\end{figure}

\section*{Prefix}
Krylov subspace methods have been applied successfully to solve various problems 
in Numerical Linear Algebra. The Netherlands have been a pioneer country in the developing of Krylov methods over the years.
Methods like the Conjugate Gradient Squared (CGS), Bi-Conjugate Gradient Stabilized (BiCGSTAB), Nested GMRES (GMRESR), and the Induced Dimension Reduction method (IDR) are examples of Krylov methods developed at Dutch universities. 
We are organizing the Student Krylov Day 2015 at TU Delft in the framework of the SIAM Student Chapter Delft. 
\newpage
\thispagestyle{empty}
\null
\vspace{10cm}
\textbf{Sponsors:}
\begin{figure}[h]
 \includegraphics[height=3cm]{Society_for_Industrial_and_Applied_Mathematics_(logo).png} \hfill
  \includegraphics[height=3cm]{TU_d_line_P1_color-TU-Delft.png} \hfill
   \includegraphics[height=3cm]{SIAMSC_Delft.png}
\end{figure}

\newpage
\section*{Program}
The Student Krylov Day takes place on February 2nd, 2015 at Technische Universiteit Delft, 
    Faculteit Elektrotechniek, Wiskunde en Informatica. We meet at \textbf{Snijderszaal LB 01.010.}
Mekelweg 4, 2628 CD Delft, The Netherlands. \\ 
\begin{table}[h]
\begin{tabular}{lll}
10:00 - 10:10 &  & Welcoming \\ [0.5ex]
10:10 - 10:50 & P. Sonnefeld & The story of IDR(s) \\ [0.5ex]
\hline \\ [-1.5ex]
11:00 - 11:20 & Manuel & Krylov methods for shifted linear systems \\ [0.5ex]
11:20 - 11:40 & Xian-Ming & Recent progresses in Krylov subspace methods\\ 
                        & & for solving complex symmetric linear systems\\  [0.5ex]
11:40 - 12:00 & Ian & Krylov and Matrix Balancing for fast Field \\ 
              &     & of Value Type Inclusion Regions\\  [0.5ex]
& & \hfill \small{Chairman: Reinaldo }  \\
\hline \\ [-1.5ex]
12:00 - 13:30 & & Lunch at TU Delft Sports Center \\ [0.5ex]
\hline \\ [-1.5ex]
13:00 - 13:20 & Heiko & Preconditioning of Large-Scale Saddle Point Systems\\
                    & & for Coupled Flow Problems\\ [0.5ex]
13:20 - 13:40 &J\"orn & A Krylov Subspace Approach to Modeling of \\
                     & & Wave Propagation in Open Domains\\ [0.5ex]
13:40 - 14:00 & Jing & A conjugate gradient based method for \\
                   & & frictional contact problems\\ [0.5ex]
& & \hfill \small{Chairman: Tom{\'a}{\v s}} \\
\hline \\ [-1.5ex]
14:30 - 14:50 & Tom{\'a}{\v s} & On the numerical behaviour of the CG method\\ [0.5ex]
14:50 - 15:10 & Patrick & Krylov subspace methods for matrix equations \\
                  & & which include matrix functions\\ [0.5ex]
15:10 - 15:30 & Ana & On Low-rank Updates of Matrix Functions\\ [0.5ex]
& & \hfill \small{Chairman: Heiko}  \\
\hline \\ [-1.5ex]
16:00 - 16:20 & Reinaldo & Induced Dimension Reduction method \\
            & & to solve the Quadratic Eigenvalue Problem \\ [0.5ex]
16:20 - 16:40 & Mario & Rational Least Squares Fitting using Krylov Spaces\\ [0.5ex]
16:40 - 17:00 & Sarah & Probabilistic bounds for the matrix condition number \\
                    & & with extended Lanczos bidiagonalization\\ [0.5ex]
& & \hfill \small{Chairman: Manuel}\\
\hline \\ [-1.5ex]
17:00 - 18:00 & & Snacks \& drinks at TU Delft
\end{tabular}
\end{table}

In the evening we will go to \href{http://www.burgerz.nl/en/contact/delft}{Burgerz} (Oude Delft 113, 2611 BE, Delft). This is a nice restaurant close to the main train station. Everybody is welcome to join!

\null\vfill\eject\thispagestyle{empty}\null\vfill\eject 
% \newpage
% \section{Book of abstracts}
\includepdf[pages={1}]{abstracts/baumann_kd15.pdf}
\includepdf[pages={1}]{abstracts/gu_kd_template.pdf}
\includepdf[pages={1}]{abstracts/zwaan_kd15_template.pdf}
\includepdf[pages={1}]{abstracts/weichelt_abstract_KrylovDay_2015.pdf}
\includepdf[pages={1}]{abstracts/kd_jtzimmerling.pdf}
\includepdf[pages={1}]{abstracts/jing_kd_template.pdf}
\includepdf[pages={1}]{abstracts/kd15_gergelits.pdf}
\includepdf[pages={1}]{abstracts/abstract_kuerschner.pdf}
\includepdf[pages={1}]{abstracts/susnjara_template.pdf}
\includepdf[pages={1}]{abstracts/kd_template_Reinaldo.pdf}
\includepdf[pages={1}]{abstracts/kd_berljafa.pdf}
\includepdf[pages={1}]{abstracts/kd15_SarahGaaf.pdf}
\newpage
\large{\textbf{Email Directory}}
\begin{itemize}
\item Reinaldo Astudillo, Delft University of Technology. \href{mailto:R.A.Astudillo@tudelft.nl}{R.A.Astudillo@tudelft.nl}.
\item Manuel Baumann, Delft University of Technology. \href{mailto:M.M.Baumann@tudelft.nl}{M.M.Baumann@tudelft.nl}.
\item Mario Berljafa, University of Manchester. \href{mailto:mario.berljafa@manchester.ac.uk}{mario.berljafa@manchester.ac.uk}.
\item Sarah Gaaf, Eindhoven University of Technology. \href{mailto:s.w.gaaf@tue.nl}{s.w.gaaf@tue.nl}.
\item Tom{\'a}{\v s} Gergelits, Charles University in Prague. \href{mailto:gergelits@karlin.mff.cuni.cz}{gergelits@karlin.mff.cuni.cz}.
\item Xian-Ming Gu, Rijksuniversiteit Groningen and University of
Electronic Science and Technology of China. \href{mailto:x.m.gu@rug.nl}{x.m.gu@rug.nl}
\item Patrick K\"{u}rschner, Max Planck Institute for Dynamics of Complex
Technical Systems Magdeburg. \href{mailto:kuerschner@mpi-magdeburg.mpg.de}{kuerschner@mpi-magdeburg.mpg.de}.
\item Peter Sonneveld, Delft University of Technology. \href{mailto:P.Sonneveld@tudelft.nl}{P.Sonneveld@tudelft.nl}.
\item Ana {\v S}u{\v s}njara, \'{E}cole Polytechnique F\'ed\'erale de Lausanne. \href{mailto:ana.susnjara@epfl.ch}{ana.susnjara@epfl.ch}.
\item Heiko Weichelt, Max Planck Institute for Dynamics of Complex
Technical Systems Magdeburg. \href{mailto:weichelt@mpi-magdeburg.mpg.de}{weichelt@mpi-magdeburg.mpg.de}.
\item Jing Zhao, Delft University of Technology. \href{mailto:J.Zhao-1@tudelft.nl}{J.Zhao-1@tudelft.nl}.
\item J\"orn Zimmerling, Delft University of Technology. \\ \href{mailto:jtzimmerling@gmail.com}{jtzimmerling@gmail.com}.
\item Ian Zwaan, Eindhoven University of Technology. \href{mailto:i.n.zwaan@tue.nl}{i.n.zwaan@tue.nl}.
\end{itemize}
\bigskip
We will tweet about the workshop using the account \texttt{@SSC\_Delft} and hashtag \texttt{\#KD15}.
\end{document}
