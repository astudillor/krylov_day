\documentclass[12pt,a4]{article}
\parindent=0.0pt
%
\begin{document}
\setcounter{page}{4}
\begin{center}
{\Large IDR-CGS-BiCGSTAB-IDR($s$)\\- a case of serendipity -\\
\mbox{}\\}
{\em Peter Sonneveld}
\footnote{Delft University of Technology,
Delft Institute of Applied Mathematics, Mekelweg 4, 2628 CD, The Netherlands.
E-mail: {\tt P. Sonneveld, p.sonneveld@tudelft.nl}}
\end{center}
In about 1976, I was preparing a renovation of the elementary course 
on numerical analysis in Delft University. In relation to the problem 
of solving a single nonlinear equation iteratively, I wondered whether 
the so-called `secant method' could be generalized to systems of $N$ nonlinear 
equations with $N$ unknowns.
\par
Before starting to read everything on a subject, I always try 
to think about it unbiased, and so I started with (probably) re-inventing the 
wheel. 
Had I seen the book by Ortega and Rheinboldt at that 
time, CGS, BiCGSTAB and IDR(s) probably wouldn't exist today. 
After a week of rather primitive numerical experiments, I decided that
generalisations of the secant method to $N$ dimensions were far too
complicated for an elementary course. However, the experiments 
showed a surprising phenomenon, that appeared to be useful in the 
machinery of solving large sparse nonsymmetric {\em linear} systems. 
\par
The first application of this `new wheel' was called IDR (Induced 
Dimension Reduction). Afterwards, CGS (Conjugate Gradients Squared)
was developed as an `improvement' of IDR, and also for other reasons.
From then, starting with BiCGStab in cooperation with Henk van der Vorst, 
a lot of other methods of this kind
were developed by many others. This went on until about 10 years ago.  
\par
In this short presentation I'll give a reconstruction
of the strange history of these so-called 
`Lanczos-type product methods'.
It will be explained why this `sleeping theory' woke up just after
my retirement in 2006,
resulting in a brand new family of methods: IDR($s$).
Since history is a continuing story, also some recently discovered 
interesting features of the IDR($s$) methods are already part of it.
Some will be mentioned in the lecture.    

\end{document}
