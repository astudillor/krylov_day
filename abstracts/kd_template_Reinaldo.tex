\documentclass{article}
\usepackage[affil-it]{authblk}
\usepackage{amsfonts}
\newcommand{\vv}[1]{\mathbf{#1}}
\title{Induced Dimension Reduction method to solve the Quadratic Eigenvalue Problem}
\author{Reinaldo Astudillo\thanks{Joint work with M. B. van Gijzen}}
\affil{PhD student at TU Delft, The Netherlands}
\date{ }

\begin{document}
\maketitle
\setcounter{page}{14}
    The Induced Dimension Reduction method (IDR($s$)) was
    originally proposed for solving systems of linear equations, and recently adapted 
    to solve the standard eigenvalue problem. In this talk,  I am going to present an extension of IDR($s$) to solve the Quadratic Eigenvalue Problem (QEP) 
    $$(\lambda^2 M + \lambda D + K)\vv{x} = \vv{0},$$
    where $M,\, D,$ and $K$ are given matrices of order $n$. Using the short-recurrences formulas of IDR, we obtain a Hessenberg decomposition to approximate eigenvalues and eigenvectors 
    of the linearized QEP. Also, exploiting the structure of the Krylov subspace vectors, we reduced the memory consumption of the proposed algorithm in almost a half.
    Numerical results generated by IDR for QEP  are competitive with respect to another specialized algorithm  
    like Second Order Arnoldi.



\end{document}
